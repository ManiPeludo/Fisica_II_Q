%----------------------------------------------------------
% Definicion del entorno 'sabermas'
\makeatletter
\definecolor{shadecolor}{rgb}{0.89,0.91,0.94}
\newenvironment{sabermas}[1]{%
\vfill
\begin{shaded}
  \begin{center}
  {\textsection{Para saber m\'as}}
  \end{center}
  #1
\sf } 
{%
\end{shaded}%
}
\makeatother

%----------------------------------------------------------
% Definicion del entorno 'problema'
\newcounter{ContadorProblema}
\setcounter{ContadorProblema}{0}
\newcounter{TieneFiguraAsociada}
\setcounter{TieneFiguraAsociada}{0}
\newcounter{UbicacionFigura}
\setcounter{UbicacionFigura}{0}

\newenvironment{problema}[2][]
{%
    \ifx\relax#1\relax%
        \setcounter{TieneFiguraAsociada}{0}
        \else
        \setcounter{TieneFiguraAsociada}{1}
    \fi
    \def \archivofigura {#1}
    % 
    \refstepcounter{ContadorProblema}
    \noindent%
    \ifnum\value{TieneFiguraAsociada} < 1%
        {\sffamily \bfseries Problema \arabic{ContadorProblema}.}
        %{\sc {#1}}%
        \par\nobreak\par\nobreak%
        \medskip 
    \else
        % Va con figura; resta determinar de que lado.
        \ifnum\value{UbicacionFigura} < 1
            % Poner la figura del lado derecho
            \begin{minipage}{12.25cm}
            {\sffamily \bfseries Problema \arabic{ContadorProblema}.}
            %{\sc {#1}}%
            \par\nobreak\par\nobreak%
            \medskip 
        \else
            % Poner la figura del lado izquierdo
            \begin{minipage}{4.5cm}
                \centering
                \includegraphics[width=4.5cm]{\archivofigura}
                {\footnotesize {\sffamily Esquema asociado al 
                problema \arabic{ContadorProblema}}.}
            \end{minipage}\hfill%
            \begin{minipage}{12.25cm}
                {\sffamily \bfseries Problema \arabic{ContadorProblema}.}
                %{\sc {#1}}%
                \par\nobreak\par\nobreak%
                \medskip 
        \fi
    \fi
}
{%
    \ifnum\value{TieneFiguraAsociada} < 1%
        % \par \bigskip \vskip 0.3cm
    \else
        % Va con figura; resta determinar de que lado.
        \ifnum\value{UbicacionFigura} < 1
            % Poner la figura del lado derecho
            \end{minipage}\hfill%
            \begin{minipage}{4.5cm}
                \centering
                \includegraphics[width=4.5cm]{\archivofigura}
                {\footnotesize {\sffamily Esquema asociado al 
                problema \arabic{ContadorProblema}}.}
            \end{minipage}
        \else
            % Poner la figura del lado izquierdo
            \end{minipage}%
        \fi
    \fi
    \setcounter{TieneFiguraAsociada}{0}
    \par \bigskip \vskip 0.3cm
    % Permutamos el valor de la ubicacion
    \ifnum\value{UbicacionFigura} < 1
        \setcounter{UbicacionFigura}{1}
    \else
        \setcounter{UbicacionFigura}{0}
    \fi
}

%----------------------------------------------------------
% Definicion/Redefinicion de estilos
\renewcommand{\vec}[1]{\ensuremath{\mathbf{#1}}}

