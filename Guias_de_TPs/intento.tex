
Guía 2: Conductores, Capacidad, Condensadores, Dieléctricos, Polarización, Campos E y D.

FALTO ESTE PROBLEMA DE LA GUIA 2

 
Problema 3:
En un campo eléctrico uniforme cargado con carga total Q.

E0    se introduce un cuerpo conductor de forma arbitraria
 
a)	¿Qué valor tiene la fuerza eléctrica que se ejerce sobre el cuerpo?
b)	Como consecuencia de la inducción de cargas sobre la superficie del conductor, el campo dejará de ser uniforme en la vecindad del cuerpo. Si se "congela'' la distribución superficial de carga y se quita el campo externo, ¿cómo será el campo en el interior del cuerpo? Notar que al congelar la carga superficial, el cuerpo pierde las propiedades de un conductor.


---------------------------------------------------------------


Guía 5: Corrientes Variables, ley de Faraday, ley de Lenz, coeficientes de inducción, energía magnética, períodos transitorios.

Problema 1:
Una espira circular de 1000 vueltas y 100 cm2 de área está colocada en un campo magnético uniforme de 0.01 T y rota 10 veces por segundo en torno de uno de sus diámetros que es normal a la dirección del campo. Calcular:
a)	la f.e.m. inducida en la espira en función del tiempo t y, en particular, cuando su normal forma un ángulo de 45° con el campo.
b)	la f.e.m. máxima y mínima y los valores de t para que aparezcan estas f.e.m.



Problema 2:
En la figura se muestra un disco de Faraday, consistente en un disco de cobre de radio a cuyo eje es paralelo a un campo magnético uniforme B .
Si  el  disco  rota  con  una  velocidad  angular  ω,  calcular  la  f.e.m.  que
aparece entre los puntos A y C.



Problema 3:
Los rieles de una vía están separados por 1.5 m y están aislados entre sí. Se conecta entre ellos un milivoltímetro. ¿Cuánto indica el instrumento cuando pasa un tren a 200 km/h? (Considere que esto pasa en Francia o en Alemania donde tal fenómeno es posible). Suponer que la componente vertical del campo magnético de la Tierra mide allí  1.5×10-5 T.



Problema 4:
Un cable rectilíneo muy largo conduce una corriente I de 1A. A una distancia L = 1 m del cable se encuentra el extremo de una aguja de 40 cm de largo que gira en torno de ese extremo en el plano del cable, con una velocidad angular ω = 20π s-1, como se muestra en la figura. Calcular la f.e.m. inducida en los extremos de la aguja como función del tiempo.



Problema 5:
Un solenoide tiene 1000 vueltas, 20 cm de diámetro y 40 cm de largo. En su centro se ubica otro solenoide de 100 vueltas, 4 cm de diámetro y espesor despreciable, cuya resistencia vale 50 Ω. Si la corriente que circula por el solenoide exterior aumenta a razón de 0,5 A cada 0.2 s, calcular la corriente que se induce en el solenoide interior, cuya autoinductancia es de 2,4 mH.

Problema 6:
Calcular la autoinductancia de:
a)	un solenoide infinito de radio R y n vueltas por unidad de longitud (exprese el resultado por unidad de longitud).
b)	un toroide con N vueltas, sección S y radio medio R, usando que la diferencia entre el radio exterior e interior es mucho menor que R.
c)	un solenoide de longitud L y radio R (suponga R << L), con N vueltas.
 



Problema 7:
Calcule la energía magnética por unidad de longitud para el cable coaxil del Problema 10 de la Guía 4. Utilizando la relación entre la energía y la autoinductancia, encuentre esta última.

Problema 8:
Dos cables rectilíneos paralelos de radio r, separados por una distancia d, pueden suponerse como un circuito que se cierra por el infinito. Encuentre la autoinductancia por unidad de longitud cuando r << d.



Problema 9:
Calcule M12 y M21 entre una espira circular de radio R y un solenoide finito de longitud L y radio r (suponga r << L y r << R), dispuestos de tal forma que los centros y los ejes de ambos son coincidentes. Utilice las aproximaciones que crea necesarias y diga cuál de los dos resultados es más confiable cuando L es pequeño respecto a R.

Problema 10:
Dos bobinas están conectadas en serie a una distancia tal que la mitad del flujo de una de ellas atraviesa también la otra. Si la autoinducción de las bobinas es L, calcular la autoinducción del conjunto, suponiendo que las bobinas están conectadas de tal forma que los flujos se suman.



Problema 11:
Un condensador de 3 µF se carga a 271.8 V y luego se descarga a través de una resistencia de 1MΩ. Calcular:
a)	el voltaje sobre el condensador luego de 3 segundos.
b)	el calor disipado en la resistencia durante la descarga completa del condensador. Comparar el valor obtenido con la energía almacenada en el condensador al comienzo de la descarga.



Problema 12:
La figura muestra las condiciones del circuito antes de t=0, instante en que se cierra la llave S. Calcular para toda t > 0:
a)	El voltaje sobre el condensador C2.
b)	La corriente.



Problema 13:
Una f.e.m. de 400 V se conecta en tiempo t = 0 a un circuito serie formado por una inductancia
L = 2 H, una resistencia R = 20 Ω y un capacitor C = 8 µF inicialmente descargado.
a)	Demostrar que el proceso de carga es oscilatorio y calcular la frecuencia de las oscilaciones.
Comparar esta frecuencia con el valor de (LC)-1/2.
b)	Calcular la derivada temporal inicial de la corriente.
c)	Hallar, en forma aproximada, la máxima tensión sobre C.
d)	¿Qué resistencia debe agregarse en serie para que el amortiguamiento del circuito sea crítico?
 


 
Problema 14:
Considere el circuito que se muestra en la figura. Todas las resistencias son iguales. En el instante t0=0 se cierra la llave que conecta al capacitor. Calcule en cuánto tiempo a partir de t0 el capacitor habrá alcanzado el 99% de su carga máxima, suponiendo que inicialmente estaba descargado, e indique cuál será su polaridad.
Ayuda: reduzca el circuito a sólo dos mallas.
 



R2

V	R1

C
R3	R4
R6	R5
 


Guía 6: Circuitos de Corriente Alterna

Problema 1:
Un condensador C = 1µF está conectado en paralelo con una inductancia L = 0.1 H cuya resistencia interna vale R = 1Ω. Se conecta la combinación a una fuente alterna de 220V y 50Hz. Determine:
a)	la corriente por el condensador.
b)	la corriente por la inductancia.
c)	la corriente total por la fuente.
d)	la potencia total disipada.
Construir el diagrama vectorial en el plano complejo para cada paso.



Problema 2:
Una resistencia R, un condensador C y una inductancia L están conectados en serie.
a)	Calcular la impedancia compleja de la combinación y su valor en resonancia (esto es, cuando la reactancia X se anula).
b)	Construir el diagrama vectorial. Empleándolo, hallar el valor de la impedancia cuando X = R y para la resonancia. Notar que existen dos valores de frecuencia (ω2 y ω1) para los cuales se tiene X = R.
c)	Trazar la curva de resonancia y hallar el ancho de banda (ω2 - ω1).
d)	Repetir los puntos anteriores suponiendo ahora que los mismos componentes se conectan
en paralelo.



Problema 3:
Tres impedancias Z1, Z2, y Z3  están conectadas en paralelo a una fuente de 40V y 50Hz. Suponiendo que  Z1 = 10Ω, Z2 = 20 (1+j) Ω y Z3 = (3−4j) Ω:
a)	calcular la admitancia, conductancia y susceptancia en cada rama.
b)	calcular la conductancia y la susceptancia resultante de la combinación.
c)	calcular la corriente en cada rama, la corriente resultante y la potencia total disipada.
d)	trazar el diagrama vectorial del circuito.



Problema 4:
Una inductancia L que tiene una resistencia interna r está conectada en serie con otra resistencia R = 200 Ω. Cuando estos elementos están conectados a una fuente de 220 V y 50Hz, la caída de tensión sobre la resistencia R es de 50V. Si se altera solamente la frecuencia de la fuente, de modo que sea 60 Hz, la tensión sobre R pasa a ser 44 V. Determinar los valores de L y r.



Problema 5:
En el circuito indicado, la fuente de tensión E entrega 100V con una frecuencia de 50Hz y los elementos que lo constituyen son:
C = 20 µF, L = 0.25 H, y R1 = R2 = R3 = 10 Ω.
a)	Calcular la impedancia equivalente a la derecha de los puntos A y B
b)	Calcular la corriente que circula por cada resistencia.
c)	Construir el diagrama vectorial del circuito.
 




Problema 6:
Para el circuito de la figura:
a)	hallar el valor de la impedancia compleja equivalente.
b)	determinar su valor en resonancia.
c)	¿cuánto vale la frecuencia ω en este caso?
d)	construir el diagrama vectorial de la corriente por cada una
de las ramas.



Problema 7: (Optativo)
Para el circuito de la figura, hallar:
a)	las	corrientes	que	circulan	por	cada	rama empleando el método de mallas.
b)	la potencia suministrada por cada generador.
c)	la potencia disipada en cada impedancia.
Datos: V1  = 30V, V2  = 20 V, Z1  = 5 Ω, Z2  = 4 Ω,  Z3  = (2+3j) Ω, Z4  = 5j Ω, Z5  = 6  Ω y f = 50Hz.

