\documentclass[problemas]{guia}

\def \practnum {1} 
\def \practica {Electrost\'atica}

\def \materia {Laboratorio de F\'\i sica II para Qu\'\i micos}
\def \periodo {2\sptext{o} cuatrimestre de 2016}
\def \profesor {Diana Skigin}
\def \website {http://materias.df.uba.ar/f2qa2016c2}
 
\usepackage{graphics}
\usepackage{amsmath}
\usepackage{amsfonts}
\usepackage{graphicx}
\usepackage{float}
\usepackage{wrapfig}
\usepackage{subfigure}
\usepackage{bm}
\usepackage{grffile}
\usepackage{color}
\usepackage{framed}
\usepackage[utf8]{inputenc}
\usepackage[T1]{fontenc}
\usepackage{lmodern} 
\usepackage{enumerate}
% definicion del entorno 'sabermas'
\makeatletter
\definecolor{shadecolor}{rgb}{0.89,0.91,0.94}
\newenvironment{sabermas}[1]{%
\vfill
\begin{shaded}
  \begin{center}
  {\textsection{Para saber m\'as}}
  \end{center}
  #1
\sf } 
{%
\end{shaded}%
}
\makeatother

\renewcommand{\vec}[1]{\ensuremath{\mathbf{#1}}}



\hyphenation{ coe-fi-cien-tes coe-fi-cien-te au-to-va-lor
              au-to-va-lo-res co-rres-pon-der pro-ble-ma 
              cual-quie-ra po-la-ri-za-cio-nes }


\begin{document} 
\maketitle

\begin{problema}[]{}
    \begin{enumerate}[(a)]
        \item Calcular el cociente $q/m$ entre la carga y la masa de dos 
            partículas idénticas que se repelen electrostáticamente con la 
            misma fuerza con que se atraen gravitatoriamente. Comparar el valor
            hallado con el cociente $e/m$ para el electrón. \\

            Datos: $G = 6.7\times10^{11}$ N m$^2$/kg$^2$;  
            $k = 9\times10^9$ N m$^2$/C$^2$ ;  $m_e = 9.11\times10^{-31}$ kg ;
            $e = 1.6\times10^{-19}$ C. \\

        \item Calcular la fuerza gravitatoria entre dos esferas de 1 cm de 
            diámetro, de cobre, separadas una distancia de 1 m. Si se retirara
            a cada esferita un electrón por átomo, ¿cuál sería la fuerza de 
            repulsión electrostática entre ambas? \\

            Datos: $\delta_{Cu} = 9$ g/cm$^3$; $N_A = 6.02\times10^{23}$; 
            $A_{Cu} = 63.5$.
    \end{enumerate}
\end{problema}

\begin{problema}[ctanlion.png]{}
    En la figura se muestran tres cargas puntuales idénticas, cada una de masa 
    $m = 0.100$ kg y carga $+q$, colgadas de tres cuerdas. Si la longitud de 
    las cuerdas izquierda y derecha es $L = 30$ cm y el ángulo 
    $\theta = 45^\circ$, determine el valor de $q$ sabiendo que el sistema se 
    encuentra en equilibrio.
\end{problema}

\begin{problema}[ctanlion.png]{}
    Tres cargas puntuales están ubicadas en los vértices de un triángulo 
    equilátero de $0.5$ m de lado, como indica la figura. Calcule la fuerza 
    eléctrica neta sobre la carga de $7~\mu$C.
\end{problema}

\begin{problema}[ctanlion.png]{}
    Cuatro cargas puntuales idénticas ($q=+10~\mu$C) se localizan en las 
    esquinas de un rectángulo, como se indica en la figura. Las dimensiones 
    del rectángulo son $L = 60$ cm y $D = 15$ cm. Calcule la magnitud y 
    dirección de la fuerza eléctrica neta ejercida sobre la carga en la esquina
    izquierda inferior por las otras tres cargas.
\end{problema}

\begin{problema}{} 
    Un dipolo eléctrico puede suponerse compuesto por una carga positiva $q$ y
    otra negativa $-q$ separadas una distancia $2a$, como se aprecia en la 
    figura. Determine el campo eléctrico $\vec{E}$ debido a estas cargas a lo 
    largo del eje y en el punto $\vec{P} = (0,y)$. Suponga que $y$ es mucho 
    mayor que $a$. Repita el cálculo para un punto sobre el eje $x$.
\end{problema}

\begin{problema}{} 
    Halle la fuerza neta sobre una carga $q$ ubicada en el centro de un 
    cuadrado de lado $L$, cuando se han colocado cargas $q$, $2q$, $4q$ y $2q$
    en los cuatro vértices (en ese orden). Saque provecho de la simetría de la
    configuración de cargas para simplificar el cálculo.
\end{problema}
 
\begin{problema}{} 
    Dos cargas puntuales idénticas $+q$ están fijas en el espacio y separadas 
    por una distancia $d$. Una tercera carga $-Q$ puede moverse libremente y 
    se encuentra inicialmente en reposo donde muestra la figura, con 
    coordenadas $(x,0)$, a igual distancia de ambas cargas $+q$. Muestre que 
    si $x$ es pequeña en relación con $d$, el movimiento de $-Q$ es armónico 
    simple a lo largo de la recta que equidista de ambas cargas $+q$ y 
    determine el período de ese movimiento.
\end{problema}

\begin{problema}{} 
    En dos vértices contiguos de un cuadrado de lado $L$ se hallan dos cargas 
    $q$. En los dos vértices restantes se colocan dos cargas $-q$. Determine, 
    empleando razonamientos de simetría, cuál será la dirección y el sentido 
    del campo eléctrico sobre los ejes perpendiculares a los lados del cuadrado
    por el punto medio de los mismos. Calcule el campo eléctrico sobre dichos 
    ejes.
\end{problema}

\begin{problema}{} 
    Un hilo muy fino de longitud $L$ está cargado uniformemente con una carga 
    total $Q$. Calcular el campo eléctrico sobre el plano medio del hilo.
\end{problema}

\begin{problema}{} 
    Una corona circular de radios $a$ y $b$ tiene una densidad de carga 
    uniforme $\sigma$.
    \begin{enumerate}[(a)]
        \item Hallar el campo eléctrico en su eje. 
        \item Deducir del resultado anterior el campo eléctrico en el eje de 
            un disco de radio $b$ y luego el campo eléctrico de un plano, ambos
            cargados uniformemente. En cada caso estudie la continuidad del 
            campo y obtenga el valor del {\it salto} en la discontinuidad.
    \end{enumerate}
\end{problema}

\begin{problema}{} 
    En cada uno de los casos siguientes determine, explotando la simetría de 
    la configuración de cargas, cuál será la dirección del campo eléctrico y 
    de cuáles coordenadas dependerán sus componentes. Utilizando el teorema de
    Gauss determine el campo eléctrico en todo el espacio, y a partir de éste 
    calcule el potencial electrostático. Grafique las líneas de campo y las 
    superficies equipotenciales.
    \begin{enumerate}[(a)]
        \item Un hilo delgado infinito con densidad lineal uniforme $\lambda$.
        \item Un cilindro circular infinito de radio $R$, cargado uniformemente
            en volumen con densidad $\rho$.
        \item Un plano infinito con densidad superficial de carga uniforme 
            $\sigma$.
        \item Una esfera de radio $R$ con densidad uniforme $\rho$.
        \item Una esfera de radio $R$ con densidad de carga $\rho = A \: r^n$,
            con $A$ y $n$ constantes.
    \end{enumerate}

    Nota: Observe que en los tres primeros casos no se puede tomar el cero de 
    potencial en el infinito ni se lo puede calcular mediante la integral:
    \begin{equation*}
        V(\vec{r}) = k \int \frac{\rho(\vec{r}')}{|\vec{r}-\vec{r}'|} \: % 
        d^3 \vec{r}' + \text{constante} 
    \end{equation*}
    ya que ella no está definida para esas distribuciones de carga.
\end{problema}
 
\begin{problema}{} 
    Calcule la integral definida en el problema anterior para la situación 
    descripta en el Problema 9. Verifique que su gradiente es $-\vec{E}$. ¿Qué
    ocurre cuando la longitud del hilo se hace infinita? \\
    
    Nota: Dado que estamos calculando el potencial sólo para puntos sobre un 
    plano perpendicular al hilo y que pasa por el centro del mismo, el 
    resultado no sirve para obtener la componente del campo  eléctrico  
    perpendicular  a  ese  plano.  Sin  embargo,  por  simetría  sabemos  que
    esa componente debe ser nula.
\end{problema}

\begin{problema}{} 
    En ciertas condiciones, el campo eléctrico de la atmósfera apunta hacia la
    superficie de la Tierra. Sobre la superficie su valor es de 300 V/m, 
    mientras que a 1400 m de altura, es de 20 V/m.
    \begin{enumerate}[(a)]
        \item Calcule la carga total contenida en un volumen cilíndrico 
            vertical cuya base está sobre la superficie terrestre y su altura 
            es de 1400 m. ¿Cuál es la carga media por unidad de volumen en esa
            región de la atmósfera? (Suponga que el problema es plano). 
        \item En la atmósfera podemos encontrar iones negativos y positivos. 
            Suponiendo que el valor absoluto de la carga de cada ion es 
            $e = 1.6\times10^{-19}$ C, escriba la densidad de carga como 
            función de $n_-$ y $n_+$ (número de iones negativos y positivos 
            por unidad de volumen, respectivamente).
    \end{enumerate}
\end{problema}

\begin{problema}{} 
    Una molécula de agua tiene su átomo de oxígeno en el origen y los núcleos 
    de hidrógeno en $x = ( \pm 0.077$ nm; $0.058$ nm). Si los electrones del 
    hidrógeno se transfieren completamente al átomo de oxígeno, ¿cuál sería el
    momento dipolar de la molécula? Compare con el valor experimental (esta 
    caracterización de los enlaces químicos del agua como totalmente iónicos 
    sobrestima el momento dipolar).
\end{problema}

\begin{problema}{} 
    Un anillo de radio $R$ está cargado uniformemente con una carga total $-q$.
    En el centro del mismo se coloca una carga puntual $q$.
    \begin{enumerate}[(a)]
        \item ¿Cuánto valen los momentos monopolar y dipolar? ¿Depende el 
            momento dipolar del origen de coordenadas?
        \item Calcule el potencial y el campo eléctrico sobre el eje del anillo
            y estudie el comportamiento a distancias grandes.
    \end{enumerate}
\end{problema}

\end{document}
