\documentclass{article}
% \usepackage{a4wide}
\usepackage{hyperref}% http://ctan.org/pkg/hyperref
\AtBeginDocument{%
  \let\oldref\ref% 
  \def\ref{\oldref*}}

\begin{document}
\title{La luz, es una onda? (1803)}
\author{P. Cobelli}
\date{Fecha de \'ultima actualizaci\'on: 27 de Octubre de 2015}
\maketitle

La naturaleza de la luz es uno de los principales problemas que los 
cient\'\i ficos de fines del siglo XVII y principios del XVIII intentaron
resolver. En el siglo XVIII, la mayor\'\i a de los f\'\i sicos optaban,
junto a Newton, por una teor\'\i a corpuscular de la luz. No obstante, algunos
de sus contempor\'aneos, como Robert Hooke, Christian Huygens y Francesco
Grimaldi, comienzan a entrever que la luz pueda estar compuesta de ondas, 
aunque esta noci\'on no se corresponda exactamente con el sentido que hoy
le damos. Les faltaba, sin embargo, una prueba experimental, que Thomas
Young (el h\'eroe de nuestra \'ultima clase te\'orica) vendr\'\i a a dar
en 1803. 

Thomas Young fue indudablemente uno de los grandes esp\'\i ritus de comienzos
del siglo XIX. Fil\'ologo confirmado (hablaba seis idiomas y tuvo, mucho
antes que Champollion, ideas brillantes vinculadas al descifrado de los 
jerogl\'\i ficos egipcios), era tambi\'en un apasionado de la bot\'anica, 
la filosof\'\i a, la qu\'\i mica y la medicina; antes de volverse hacia la
f\'\i sica y, m\'as particularmente, hacia {\it el problema de la luz}. 
Young dice en sus cuadernos al respecto: 

\begin{quote}
    {\it
`[...] el problema de la luz, una cuesti\'on sin importancia desde el punto de
    vista de la vida cotidiana o profesional, pero extremadamente interesante
    en la medida en que nos permite comprender la naturaleza de nuestros 
    sentidos y la constituci\'on del universo en general. [...]'
}
\end{quote}

La primera experiencia de Young consistir\'a en hacer incidir la luz solar
sobre el canto de un naipe, y a estudiar el patr\'on de sombras que \'esta
proyecta. Young observa entonces la aparici\'on de una serie de franjas que
\'el juzga causadas por la difracci\'on\footnote{} de la luz por el 
obst\'aculo:

\begin{quote}
    {\it
`[...] Estas franjas son debidas a los efectos conjuntos de las fracciones
de luz que pasan a uno y otro lado del naipe; luz que es torcida, o m\'as
correctamente, difractada (partida) por \'el. [...]'
    }
\end{quote}

Este fen\'omeno de difracci\'on, en el cu\'al la luz parece `rodear' los
obst\'aculos, constituye una primera prueba de su naturaleza ondulatoria, 
dado que una `luz corpuscular' deber\'\i a necesariamente ser detenida por
un obst\'aculo sin poderlo rodear. 

En una experiencia a\'un m\'as decisiva, Young har\'a pasar un haz de luz
a trav\'es de dos rendijas delgadas talladas por \'el mismo sobre una pantalla.
En lugar de observar una \'unica mancha luminosa, la combinaci\'on de las
dos luces produce una serie de rayas paralelas, unas brillantes y otras 
oscuras, dispuestas en forma equidistante unas de otras. Estas son las 
hoy c\'elebres franjas de interferencia (que vimos la \'ultima clase).

Young va a dar una explicaci\'on a este fen\'omeno extra\~no por el que la
luz es capaz de crear oscuridad. Cuando la luz atraviesa las rendijas de su
dispositivo, esta se desdobla en dos ondas esf\'ericas coherentes. Ahora bien,
los trayectos recorridos por la luz hasta la pantalla son diferentes: 
existe una `diferencia de camino' entre ellos. Las dos ondas est\'an entonces
alternativamente en fase, en cuyo caso se refuerzan mutuamente produciendo
una franja brillante: estas son las interferencias constructivas. Por otro 
lado, cuando est\'an en contrafase, se anulan mutuamente y dan lugar a 
una franja oscura: una interferencia destructiva.

Como dijimos en las clases te\'oricas, hay que notar que la interferencia se
genera en todos los fen\'omenos ondulatorios\footnote{
Siempre que las ondas 
intervinientes presenten {\it coherencia}. 
} 
(en ondas de sonido, en ondas
en la superficie libre de un l\'\i quido, etc.) era ya conocida por Young, 
quien nota que {\it `los sonidos
musicales consisten en cualidades opuestas, capaces de neutralizarse 
mutuamente; por esto podr\'\i amos concluir que debe existir una fuerte
similitud entre la naturaleza del sonido y de la luz'}. En suma, seg\'un
Young, la luz es de naturaleza ondulatoria, tal y como el sonido.

Convencido de la solidez de su an\'alisis, Young llega incluso a medir 
la longitud de onda de la luz (estas son las primeras mediciones de la
longitud de la luz de la historia); dice Young: 

\begin{quote}
    {\it 
        `[...] De una comparaci\'on 
entre las diversas experiencias, resultar\'\i a que la longitud de las
oscilaciones que constituyen la luz roja debe ser, en el aire, del 
orden de 36/1000 de pulgada; y las de la luz violeta, de aproximadamente
60/1000 de pulgada; la media del espectro luminoso total siendo de
aproximadamente 45/1000 de pulgada. [...]' 
}
\end{quote}

Por desgracia Young no precisa 
qu\'e t\'ecnica o montaje experimental emple\'o para obtener estos 
resultados, lo que es todav\'\i a m\'as lamentable dado que sus valores
num\'ericos est\'an muy cerca (dentro del error experimental) de los
aceptados en la actualidad.


\section*{Para saber m\'as}

Para conocer m\'as acerca de este tema (y de otros trabajos de Thomas Young)
les recomiendo fuertemente la lectura directa de la fuente original.
Los trabajos completos de Young fueron reunidos en un compendio que 
consta de dos vol\'umenes, y publicado por vez primera en 1807. 
Esta obra esta disponible en l\'\i nea, tanto para leer on-line
como para descargar en varios formatos (PDF, ePUB, etc.), en los
siguientes links (los links son clickeables!):

\begin{itemize}
    \item `A Course of Lectures on Natural Philosophy and the Mechanical Arts',
        Vol. 1. \url{https://archive.org/details/lecturescourseof01younrich}
    \item `A Course of Lectures on Natural Philosophy and the Mechanical Arts',
        Vol 2. \url{https://archive.org/details/lecturescourseof02younrich}
\end{itemize}

Las tem\'aticas asociadas a la naturaleza ondulatoria de la luz y la
descripci\'on de sus resultados experimentales las encontrar\'an en el
segundo volumen. 

Les recomiendo tambi\'en hojear (aunque s\'olo sea eso) los esquemas
y diagramas [que se encuentran al final de cada volumen]; comprobar\'an 
-no sin cierto estupor- que
su calidad es muy superior a la gran mayor\'\i a de las figuras que 
hoy encontramos en publicaciones y/o libros de texto.

\end{document}
