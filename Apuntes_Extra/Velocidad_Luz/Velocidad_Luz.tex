\documentclass{article}
% \usepackage{a4wide}
\usepackage{hyperref}% http://ctan.org/pkg/hyperref
\AtBeginDocument{%
  \let\oldref\ref% 
  \def\ref{\oldref*}}

\begin{document}
\title{Acerca de la velocidad de la luz}
\author{P. Cobelli}
\date{Fecha de \'ultima actualizaci\'on: 29 de Octubre de 2015}
\maketitle

\section{La velocidad de la luz es finita (1676)}

La pregunta acerca de la velocidad de la luz puede hallarse ya en los inicios
de la f\'\i sica. Los autores griegos que la evocan est\'an (en su gran 
mayor\'\i a) convencidos de la instantaneidad con la que la luz se 
transmite a trav\'es del espacio. 

Este hecho no debe sorprendernos, dado que el fen\'omeno de la propagaci\'on
de la luz es imperceptible a nuestros sentidos. En efecto, la luz se propaga
a aproximadamente 300 000 km/s (el valor moderno es de 299 792,458 km/s).
En este sentido, podemos tomar las palabras del historiador de ciencias G.
Picolet:

\begin{quote}
    {\it 
        `[...] el espacio visual ordinario en el cual nosotros vivimos en
        la Tierra no excede casi algunas decenas, quiz\'as centenas de 
        kil\'ometros. En consecuencia, el tiempo empleado por la luz para
        propagarse es demasiado breve, al punto de ser insensible para 
        nosotros. Por ejemplo, en una distancia de 100 km, la duraci\'on
        de la propagaci\'on lum\'\i nica es del orden de 1/3000 segundos,
        valor que es inferior al umbral diferencial de percepci\'on del
        ojo humano. [...]'
    }
\end{quote}

Por tanto, tampoco debe sorprendernos que los argumentos de los defensores
de la instantaneidad de la luz se basen en la experiencia cotidiana: la 
aparici\'on y desaparici\'on inmediata de las im\'agenes en los espejos, 
la reapaci\'on inmediata de la luz solar en tierra al despejarse el cielo,
etc. 

Ciertos autores de la antig\"uedad, concientes de las posibles limitaciones
de nuestros sentidos para discernir la propagaci\'on de la luz, avanzan 
argumentos basados en distancias astron\'omicas. Afirman, por ejemplo, 
que `al mismo tiempo que el Sol aparece sobre el horizonte, todo el 
hemisferio superior de la Tierra es iluminado instant\'aneamente' (Fil\'opono),
o que, cuando uno cierra sus ojos y los abre en direcci\'on al cielo 
estrellado, la luz de las estrellas golpea inmediatamente nuestros 
sentidos; prueba de su instantaneidad. Aquellos que, como Emp\'edocles, 
persisten en creer que la luz se transmite a velocidad finita, son llamados
al orden por autores tales como Arist\'oteles, qui\'en hace notar que 
`una opini\'on tal es contraria tanto a la raz\'on como a los hechos 
observados' (esta afirmaci\'on puede encontrarse en {\it Acerca del Alma}).

La concepci\'on instantaneista de la luz se perpet\'ua luego en la Edad Media
y en el Renacimiento; en este caso evidentemente ligada a -adem\'as del peso
de la tradici\'on- a la insuficiencia de precisi\'on del instrumental 
experimental disponible por aquellos tiempos. En este sentido cabe recordar
que el telescopio, que permite fijar puntos de referencia astron\'omica, data
de 1609; el reloj a p\'endulo, que hace posible la medida de lapsos inferiores
al segundo, reci\'en se pone a punto en 1657. De esta forma, grandes 
cient\'\i ficos tales como Johannes Kepler y Ren\'e Descartes siguen creyendo
a\'un en la instantaneidad de la luz.

Galileo, por otro lado, piensa que la luz tiene una velocidad de propagaci\'on
finita, y lo sugiere en sus {\it Discursos sobre los dos grandes sistemas
del mundo} de 1632. Su hip\'otesis ser\'a verificada al inicio de los a\~nos
1670, como consecuencia de una serie de experimentos llevados a cabo en el
Observatorio de Par\'\i s. En ellos, se observaron los eclipses de los cuatro
sat\'elites de J\'upiter, que ya eran conocidos desde 1610 (y fueron 
precisamente descubiertos por Galileo mismo).

Estos sat\'elites son, en efecto, peri\'odicamente eclipsados cada vez que
ingresan en el cono de sombra producido por J\'upiter, que se interpone entre
el Sol y ellos. Sin embargo, el astr\'onomo Giovanni Cassini hab\'\i a ya
observado irregularidades en el movimiento del primer sat\'elite, que se
manifestaban como retardos a la salida del cono de sombra (proceso que en
astronom\'\i a se conoce como {emersi\'on}) cada vez que J\'upiter entra en
conjunci\'on con el Sol y juntos se alejan de la Tierra. La otra irregularidad
detectada por Cassini estaba asociada a una entrada anticipada al cono de 
sombra ({\it imersi\'on}) cada vez que J\'upiter entra en oposici\'on con
el Sol y se acerca a la Tierra.

Cassini entrev\'e la hip\'otesis seg\'un la cu\'al las diferencias observadas
son debidas a una velocidad finita propagaci\'on de la luz. Sin embargo 
abandona r\'apidamente esta explicaci\'on, dado que no logra poner en 
evidencia este mismo fen\'omeno para los otros tres sat\'elites jovianos.

Esta hip\'otesis es luego tomada por el astr\'onomo dan\'es Olaf R\"omer,
quien concluye que si el per\'\i odo entre dos eclipses parece m\'as largo
cuando J\'upiter se aleja de la Tierra, es porque la luz debe recorrer una
distancia mayor para informarnos del comienzo del segundo eclipse. El inverso
se produce entonces, cuando J\'upiter se acerca a la Tierra seis meses m\'as
tarde: el tiempo de viaje de la luz es ahora m\'as corto, y el ingreso del
sat\'elite en el cono de sombra de J\'upiter nos parece anticiparse.

R\"omer compara los tiempos de retardo y de anticipaci\'on de los eclipses
del primer sat\'elite joviano, seg\'un J\'upiter y el Sol se encuentran en
conjunci\'on o en oposici\'on, y deduce que la velocidad de la luz es finita.
Seguro de esta afirmaci\'on, anuncia en septiembre de 1676 en la Academia 
Real de Ciencias que el eclipse del primer sat\'elite de J\'upiter, previsto
para el 9 de noviembre siguiente, se producir\'\i a con 10 minutos de retardo,
teniendo en cuenta el aumento de la distancia Tierra-J\'upiter. Su predicci\'on
es confirmada por las observaciones.

En el informe de sus trabajos, R\"omer no ofrece un valor para la velocidad de
la luz, sino que se contenta con observar (y demostrar!) que su propagaci\'on
tiene lugar a una velocidad finita. `?Por qu\'e? El astr\'onomo franc\'es
Jean-Pierre Verdet nos ofrece una posible explicaci\'on: {\it `para determinar
dicha velocidad, hac\'\i a falta disponer de la distancia en cuesti\'on. 
En esa \'epoca ese problema estaba a\'un abierto y las estimaciones 
disponibles eran muy dis\'\i miles unas de otras`}. 

\section{La primera determinaci\'on experimental (1849)}

El primer valor moderno para la velocidad de propagaci\'on de la luz fue 
obtenida en 1849 por el f\'\i sico franc\'es Hyppolyte Fizeau, utilizando 
un sistema por \'el mismo denominado `de rueda dentada'. 

Este sistema consist\'\i a en enfocar la luz


\end{document}
