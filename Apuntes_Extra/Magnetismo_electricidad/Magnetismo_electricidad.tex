\documentclass{article}
\usepackage{a4wide}

\begin{document}
\title{Explicar el magnetismo mediante la electricidad (1819-1820)}
\author{P. Cobelli}
\date{Fecha de \'ultima actualizaci\'on: 22 de Septiembre de 2015}
\maketitle

La relaci\'on entre electricidad y magnetismo era ya aparente para ciertos
f\'\i sicos del siglo XVIII y de principios del XIX, que sab\'\i an ya que
los rayos pod\'\i an magnetizar (imantar) el hierro: la electricidad era
entonces susceptible de engendrar el magnetismo. Varios cient\'\i ficos 
buscaban verificar esta relacion entre electricidad y magnetismo, de inter\'es
capital para sentar las bases de los fen\'omenos {\it electromagn\'eticos}.
Ser\'\i a Hans Christian Oersted, f\'\i sico dan\'es y profesor en la 
Universidad de Copenaghen qui\'en pondr\'\i a esta relaci\'on en evidencia,
en el marco de una serie de cursos de electricidad, galvanismo y magnetismo, 
dados durante el invierno de 1819-1820. 

Por mucho tiempo se crey\'o que el descubrimiento de Oersted hab\'\i a sido
fruto del azar, pero esta hip\'otesis fue recientemente refutada. Oersted
cre\'\i a que los fen\'omenos naturales pod\'\i an explicarse en base a una
o dos fuerzas fundamentales. Ya en 1813, Oersted predec\'\i a que un cable 
conductor ser\'\i a capaz de generar un campo magn\'etico; sin embargo su 
trabajo de profesor le imped\'\i a poner a punto una experiencia para testear
dicha posibilidad. 

En 1819, finalmente pone a punto un dispositivo experimental consistente en 
una br\'ujula ubicada en las cercan\'\i as de una {\it pila voltaica}. Esta
\'ultima consiste de un arreglo de veinte placas rectangulares de cobre
y zinc, cuya forma c\'oncava permite rellenar el espacio entre ellas con una
mezcla de agua, \'acido sulf\'urico y \'acido n\'\i trico, particularmente
conductor. De acuerdo a c\'alculos modernos, el potencial de esta pila 
habr\'\i a sido del orden de 15 volts. Oersted pone en contacto los extremos
opuestos de su pila (los bornes) empleando para ello un cable de cobre al 
que denomina {\it hilo conjuntivo} (hoy {\it hilo conductor}). 

En estas condiciones, Oersted acerca una porci\'on recta del cable a la
aguja imantada de su br\'ujula, que est\'a orientada en la direcci\'on norte-sur,
para observar que la aguja abandona su posici\'on. En particular, nota que
el polo que se encuentra bajo la parte del cable m\'as cercana al borne 
negativo de la pila se desv\'\i a hacia el oeste.

Oersted contin\'ua sus experiencias interponiendo diversos cuerpos entre
la aguja imantada y el cable: vidrio, metal, agua, piedras, resina, madera
y \'ambar (entre otras). El efecto de la desviaci\'on de la aguja sigue
produci\'endose.  

Al cabo de sus experiencias, Oersted declara que su consecuencia principal es
que "la aguja imantada es desviada de su posici\'on de equilibrio por la 
acci\'on del aparato galv\'anico [la pila el\'ectrica] y que este efecto 
se produce cuando el circuito est\'a cerrado y no cuando est\'a abierto
[es decir, cuando circula corriente]; es por haber dejado abierto el circuito
que muchos f\'\i sicos c\'elebres no tuvieron \'exito, a\~nos atr\'as, en
otras tentativas de este mismo tipo".

El trabajo de Oersted genera una gran repercusi\'on en el mundo cient\'\i fico,
dado que si bien un cierto n\'umero de f\'\i sicos est\'an convencidos del 
v\'\i nculo entre la electricidad y el magnetismo, la gran mayor\'\i a
(encabezada por Charles Augustin Coulomb) cree todav\'\i a que la electricidad
y el magnetismo son dos fen\'omenos de naturaleza totalmente distinta y no
imagina ni por un instante que la electricidad pueda engendrar el magnetismo. 
Los detractores de Oersted buscan entonces explicar sus observaciones 
avanzando la hip\'otesis {\it ad hoc} de que el cable conductor de cobre se hubiese
imantado. Conviene destacar que esta hip\'otesis es d\'ebil, dado que, hasta aquel 
momento se hab\'\i an observado propiedades magn\'eticas en los metales
\'unicamente para el acero y el hierro.

Sin embargo, esta no es la opini\'on del f\'\i sico franc\'es Andr\'e-Marie
Amp\`ere qui\'en asiste, el 4 y el 11 de septiembre de 1820, a una
reconstrucci\'on de la experiencia de Oersted, realizada por Gaspard de la Rive
y Fran\c ois Arago en la Academia de Ciencias de Paris. Amp\`ere nota en efecto
que las acciones electromagn\'eticas son transversales, dado que la aguja
imantada se dispone {\it en cruz} a la corriente. El f\'\i sico Paul Janet
dir\'a m\'as tarde al respecto: ``Ning\'un arreglo de mol\'eculas magn\'eticas,
por muy complicado que fuese, es capaz de explicar este hecho: las (inter)acciones
magn\'eticas son newtonianas, es decir que son fuerzas que se ejercen entre dos
puntos del espacio y est\'an dirigidas sobre la recta que los une; podr\'\i an
explicar atracciones y repulsiones, pero no rotaciones. Habr\'a entonces que 
renunciar a explicar la electricidad por el magnetismo; pero, sin embargo, por
qu\'e no hacer lo inverso? (y explicar el magnetismo por la electricidad?)''.

Esto es precisamente lo que piensa Amp\`ere, qui\'en presenta en la Academia de
Ciencias, el 18 de septiembre de 1820, sus resultados relativos a ``los nuevos
fen\'omenos galvanico-magn\'eticos''. En dicho trabajo, Amp\`ere anticipa el
empleo de un aparato constituido de una barra de hierro ubicado en el interior
de un cable enrollado en forma de h\'elice, por el cu\'al har\'a pasar una
corriente el\'ectrica. Amp\`ere prevee que este montaje, al que bautiza con el
nombre de {\it solenoide}, se comportar\'a como un im\'an. De hecho, Amp\`ere
logra imantar su solenoide bajo la sola influencia de la corriente el\'ectrica;
en presencia de una audiencia sorprendida. 

Dado que el solenoide se imanta en esta forma, podemos pensar que los imanes
son, ellos mismos, solenoides. En estas condiciones, dos solenoides deben poder
interactuar el uno con el otro, ejerciendo fuerzas de atracci\'on y de
repulsi\'on seg\'un la orientaci\'on de sus polos. 

El conjunto de todas estas experiencias, realizadas entre el 18 y el 25 de
septiembre de 1820 (hace 195 a\~nos, esta misma semana!) confirma las
predicciones de Amp\`ere y permite, de una vez y para siempre, ligar
logicamente la electricidad y el magnetismo. 


\end{document}
