\documentclass{article}
% \usepackage{a4wide}
\usepackage{hyperref}% http://ctan.org/pkg/hyperref
\AtBeginDocument{%
  \let\oldref\ref% 
  \def\ref{\oldref*}}

\begin{document}
\title{De qu\'e est\'an compuestas las estrellas? (1860)}
\author{P. Cobelli}
\date{Fecha de \'ultima actualizaci\'on: 1 de Noviembre de 2015}
\maketitle

En 1820, el fil\'osofo Augusto Comte escribe, en su {\it Curso de
Filosof\'\i a positiva} que permanecer\'a siempre fuera del alcance del
hombre conocer la naturaleza f\'\i sico-qu\'\i mica de las estrellas. Su 
afirmaci\'on, que es ampliamente aceptada en la \'epoca, podr\'\i a haber 
sido desmentida ya desde 1820, a\~no en el cu\'al el astr\'onomo ingl\'es
William Wollaston dispers\'o los rayos del Sol a trav\'es de un prisma,
siguiendo la t\'ecnica puesta a punto por Isaac Newton en 1669. Wollaston
observ\'o una serie de rayas oscuras en el espectro solar. Sin embargo, 
Comte ignoraba este resultado y el mismo Wollaston no ten\'\i a conciencia
de la importancia de su descubrimiento.

En 1814, el experto en \'optica Joseph Fraunhofer (a qui\'en ya mencionamos
en el curso) inventa un nuevo instrumento, el espectroscopio, con el cual 
se propone determinar los \'\i ndices de refracci\'on de diversos cristales.
Se trata concretamente de un teodolito -instrumento que hasta el momento se
hab\'\i a utilizado para relevamientos geof\'\i sicos- sobre cuya plataforma
Fraunhofer instala un prisma de vidrio flint de alta pureza, que le permite
obtener un alto grado de dispersi\'on (separaci\'on en colores) de un haz de 
luz que \'el mismo hace pasar por una rendija de ancho variable ubicada a una 
distancia de 8 metros. 

Observando la dispersi\'on de la luz solar con su espectroscopio, Fraunhofer
descubre 476 rayas oscuras en el espectro solar y mide con precisi\'on sus
posiciones respectivas. \'El cree poder atribuir la existencia de estas rayas
a las interferencias destructivas de la luz que vimos en clase. Extendiendo
sus investigaciones a la luz emitida por los planetas y las estrellas, 
Fraunhofer constata que los {\it espectros} (patrones de rayas) planetarios
son id\'enticos al espectro solar, de lo que concluye que los planetas 
reflejan la luz solar. Sin embargo, el espectro de la estrella Sirio posee
tres anchas rayas oscuras que `{\it por su apariencia no tienen ninguna
coincidencia con aquellas de la luz solar'}. Asimismo, Fraunhofer nota que
`{\it es posible observar rayas oscuras en los espectros de otras estrellas,
pero todas ellas parecen tener espectros diferentes'}. 

Las consecuencias te\'oricas de los descubrimientos de Fraunhofer son 
consideradas en 1834 por el qu\'\i mico ingl\'es Henry Fox Talbot, qui\'en
sugiere que es posible distinguir a las sustancias qu\'\i micas simplemente
mediante el estudio de sus espectros respectivos: `{\it Podr\'\i a decirse,
-escribe- que cuando el prisma muestra que una raya homog\'enea de un color
cualquiera es producida por una llama, esta raya indica la formaci\'on o la
presencia de un determinado compuesto qu\'\i mico'}. Experiencias posteriores
muestran efectivamente que dos llamas que contienen sustancias qu\'\i micas
distintas presentan espectros diferentes. Una relaci\'on directa parece 
existir, entonces, entre el espectro \'optico y la composici\'on qu\'\i mica
de un cuerpo luminoso. 

Esta relaci\'on es finalmente puesta en evidencia en 1860 por Gustav Kirchhoff
(ya mencionado en la primera parte de la materia por sus aportes al 
electromagnetismo) y Robert Bunsen. Ambos cient\'\i ficos se interesan por
un par particular de rayas oscuras presentes en el espectro solar, descubiertas
por Fraunhofer y bautizadas `rayas D'. Kirchhoff y Bunsen, luego de una
serie de experiencias, observan que las rayas D de Fraunhofer coinciden con el
doblete (las dos rayas cercanas) del espectro del sodio (Na). Ambos deciden
verificar si son id\'enticas las unas y las otras, ubicando a la entrada
de su equipo -por donde ingresa la luz solar- una llama que contiene sodio,
convencidos de que la doble raya brillante del espectro del sodio har\'a m\'as
brillantes a las del espectro solar, por simple adici\'on de las intensidades.
Sin embargo, sucede {\it lo contrario}: las rayas del espectro solar devienen
m\'as oscuras!

Kirchhoff sugiere una explicaci\'on: el sodio de la llama ha absorbido una
emisi\'on de sodio proveniente de la radiaci\'on solar, lo que explica que 
\'esta haya oscurecido el espectro solar en dichas posiciones espectrales (las
rayas D). Si se relevase correcta, la hip\'otesis de Kirchhoff tendr\'\i a 
consecuencias revolucionarias: significar\'\i a que ser\'\i a posible 
identificar los elementos qu\'\i micos en la superficie de los astros s\'olo
con la ayuda de t\'ecnicas espectrosc\'opicas. 

Continuando su an\'alisis, Kirchhoff concluye que es necesario distinguir, 
para una sustancia dada, entre espectros de emisi\'on y de absorci\'on, y que
las rayas oscuras en los espectros de todos los astros (observados desde 
la Tierra) debidas -como lo mostr\'o la experiencia con la llama de sodio- a
la absorci\'on selectiva de la luz por la atm\'osfera terrestre. Estas rayas
(y, en algunos casos, bandas) llamadas `tel\'uricas', pueden identificarse
gracias al conocimiento de la composici\'on de la atm\'osfera terrestre. Ellas
permiten dar cuenta de las rayas oscuras descubiertas por Wollaston y 
Fraunhofer en el espectro solar: son debidas a la absorci\'on, por un elemento
qu\'\i mico de nuestra atm\'osfera, del mismo elemento emitido por el Sol.

Kirchhoff confirma luego la validez de su hip\'otesis identificando cada uno
de los elementos constitutivos del Sol. A este fin, compara el espectro solar
con el espectro de emisi\'on de los diferentes elementos que existen sobre la
Tierra (aquellos de los que se ten\'\i a conocimiento entonces). Muestra 
as\'\i\  que, con alta probabilidad, el sodio, el calcio, el bario, el 
n\'\i quel, el magnesio, el zinc, el cobre y el hierro son constituyentes de
la atm\'osfera solar. Identifica, de esta forma, 463 rayas de un total de 
476 observadas. Las 13 no identificadas por Kirchhoff corresponden a un 
elemento no conocido entonces: el helio, que ser\'a descubierdo de forma 
independiente en 1868 por los astr\'onomos Jules Janssen y Joseph Lockyer.

De esta manera, Kirchhoff funda la astrof\'\i sica -el estudio de la 
composici\'on f\'\i sico-qu\'\i mica de los cuerpos celestes- y revoluciona
la astronom\'\i a gracias al instrumento de Fraunhofer. 

La explicaci\'on acerca del origen f\'\i sico
de las rayas espectrales deber\'a, no obstante, esperar unos a\~nos m\'as.
Niels Bohr ser\'a el encargado de explicarlas (y predecirlas) por medio de
su modelo cu\'antico del \'atomo, en 1913.




\end{document}
