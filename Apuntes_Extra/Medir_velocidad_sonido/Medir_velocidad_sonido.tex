\documentclass{article}
\usepackage{a4wide}

\begin{document}
\title{Medir la velocidad del sonido (1822)}
\author{P. Cobelli}
\date{Fecha de \'ultima actualizaci\'on: 24 de Octubre de 2015}
\maketitle

El hecho de que el sonido se propaga en el aire era ya conocido en la 
antig\"uedad por sabios tales como Vitruvio o Her\'on de Alejandr\'\i a.
Sin embargo, las primeras experiencias para determinar su velocidad fueron
conducidas reci\'en en el renacimiento, a cargo de Marin Mersenne y Pierre
Gassendi. Este \'ultimo obtuvo el valor de 1473 pies por segundo para la
velocidad de propagaci\'on del sonido en aire. Este valor, que resulta 
excesivo, es producto parcial del desconocimiento que en aquel tiempo se
ten\'\i a de la naturaleza del sonido: Gassendi le asigna un soporte propio
y no lo asimila a una propagaci\'on en el aire.

Otras iniciativas de este estilo tuvieron lugar durante el transcurso del
siglo XVII, bajo el impulso de f\'\i sicos de la Academia de Florencia; 
en particular de Edmond Halley, de Robert Boyle (qui\'en muestra que es la
elasticidad del aire la que permite la propagaci\'on del sonido), de 
Giovanni Cassini y de Christian Huygens. No obstante, los resultados 
experimentales resultan contradictorios y, frente a esta situaci\'on, la
Academia de Ciencias decide llevar a cabo una nueva experiencia en 1738, en
Par\'\i s. Esta experiencia involucra disparos de ca\~n\'on, que son 
intercambiados en plena noche entre el Observatorio de Paris, Montmartre, 
Fontenay-aux-Roses y Montlh\'ery. El cronometraje del tiempo que separa la 
aparici\'on del fuego (la se\~nal luminosa) y la llegada del sonido les 
permite a los investigadores asignar el valor de 333 metros por segundo a
la velocidad del sonido en aire a una temperatura de 0 $^\circ$C. 

Contrariamente a lo esperado, sin embargo, este valor para la velocidad del
sonido difiere del obtenido a partir de experiencias complementarias llevadas
a cabo en Alemania poco tiempo m\'as tarde. Todo parece indicar que las causas
de las diferencias observadas en uno y otro caso est\'an ligadas a la 
presencia de viento, y del estado higrom\'etrico de la atm\'osfera al momento
de realizar las mediciones. En consecuencia, una nueva experiencia, mucho 
m\'as rigurosa, es organizada en 1822 por la Oficina Francesa de Patrones y 
Medidas; y puesta bajo la direcci\'on de Fran\c cois Arago, asistido por 
Marie Riche de Prony. 

A fin de reducir los errores asociados a la presencia (inevitable) del viento,
los dos hombres deciden utilizar un m\'etodo que denominan `de fuego cruzado':

\begin{quote}
{\it 
``que consiste, como lo describir\'a luego Arago en su informe de la experiencia,
en producir dos sonidos iguales en un mismo instante en dos estaciones 
distantes y en observar, en cada una de ellas, el tiempo que el sonido 
proveniente de la estaci\'on opuesta demora en llegar: el viento producir\'a
entonces efectos contrarios sobre ambas velocidades, por lo que la media
entre esos dos resultados deber\'a ser exactamente la misma que en el caso
en que no hubiese viento''.}\footnote{Esta es una traducci\'on m\'\i a de la
transcripci\'on del {\it compte rendu} original de F. Arago que aparece en 
el libro `Physique et physiciens', de R. Massain (Ed. Magnard, Paris, 1982).}
\end{quote}

Por otro lado, Arago utiliza cron\'ometros mucho m\'as precisos que aquellos
empleados previamente. M\'as a\'un, el estado higrom\'etrico del aire, la 
temperatura y la presi\'on atmosf\'erica son medidas con mucho cuidado. 

Los resultados as\'\i\  obtenidos arrojan un valor de 340.88 m/s para la 
velocidad de propagaci\'on del sonido en el aire, a una temperatura de
15.9~$^\circ$C. Teniendo en cuenta la correci\'on de temperatura, esta
determinaci\'on d\'a una velocidad de 330.9 m/s a 0~$^\circ$C, en muy buen
acuerdo con la versi\'on moderna de 331.0~m/s a 0~$^\circ$C.

    
    

\end{document}
