\documentclass[laboratorio]{guia}
\def \materia {Laboratorio de F\'\i sica II para Qu\'\i micos}
\def \periodo {2\sptext{o} cuatrimestre de 2016}
\def \profesor {Diana Skigin}
\def \website {http://materias.df.uba.ar/f2qa2016c2}

\usepackage{graphics}
\usepackage{amsmath}
\usepackage{amsfonts}
\usepackage{graphicx}
\usepackage{float}
\usepackage{wrapfig}
\usepackage{subfigure}
\usepackage{bm}
\usepackage{grffile}
\usepackage{color}
\usepackage{framed}
\usepackage[utf8]{inputenc}
\usepackage[T1]{fontenc}
\usepackage{lmodern}
% definicion del entorno 'sabermas'
\makeatletter
\definecolor{shadecolor}{rgb}{0.89,0.91,0.94}
\newenvironment{sabermas}[1]{%
\vfill
\begin{shaded}
  \begin{center}
  {\textsection{Para saber m\'as}}
  \end{center}
  #1
\sf } 
{%
\end{shaded}%
}
\makeatother

\renewcommand{\vec}[1]{\ensuremath{\mathbf{#1}}}



\hyphenation{ coe-fi-cien-tes coe-fi-cien-te au-to-va-lor
              au-to-va-lo-res co-rres-pon-der pro-ble-ma 
              cual-quie-ra po-la-ri-za-cio-nes }

% \graphicspath{{./guia8/}}

\def \practnum {5}
\def \practica {Fuerza electromotriz inducida y ley de Faraday este titulo
deberia ser muy pero muy largo para que no pueda caber}
\def \practcor {Titulo corto de la guia para encabezados de pagina}

\begin{document}
\objetivo{Estudio experimental de la Ley de Inducción de Faraday usando un generador de funciones y un osciloscopio o un sistema de toma de datos con una computadora.
\tematicas{esto}}
\maketitle

\section{Experimentos}

\subsection{Bobina e imán permanente}
¿Qué sucede cuando el campo magnético generado por un imán permanente varía dentro de
una bobina como por ejemplo cuando uno acerca o mueve un imán? Conecte una bobina al
osciloscopio y registre en función del tiempo el voltaje que se induce en la misma cuando se acerca
un imán al interior de la misma. Estudie como varía el voltaje inducido de acuerdo a como se
mueve el imán. Nota: para observar una señal temporal corta conviene utilizar la función de disparo
\'unico del osciloscopio.

\begin{equation}
\vec{A} = \vec{\nabla} \phi
\end{equation}


\subsection{Bobina e imán permanente}
¿Qué sucede cuando el campo magnético generado por un imán permanente varía dentro de
una bobina como por ejemplo cuando uno acerca o mueve un imán? Conecte una bobina al
osciloscopio y registre en función del tiempo el voltaje que se induce en la misma cuando se acerca
un imán al interior de la misma. Estudie como varía el voltaje inducido de acuerdo a como se
mueve el imán. Nota: para observar una señal temporal corta conviene utilizar la función de disparo
\'unico del osciloscopio.

\subsection{Bobina e imán permanente}
¿Qué sucede cuando el campo magnético generado por un imán permanente varía dentro de
una bobina como por ejemplo cuando uno acerca o mueve un imán? Conecte una bobina al
osciloscopio y registre en función del tiempo el voltaje que se induce en la misma cuando se acerca
un imán al interior de la misma. Estudie como varía el voltaje inducido de acuerdo a como se
mueve el imán. Nota: para observar una señal temporal corta conviene utilizar la función de disparo
\'unico del osciloscopio.

\subsection{Bobina e imán permanente}
¿Qué sucede cuando el campo magnético generado por un imán permanente varía dentro de
una bobina como por ejemplo cuando uno acerca o mueve un imán? Conecte una bobina al
osciloscopio y registre en función del tiempo el voltaje que se induce en la misma cuando se acerca
un imán al interior de la misma. Estudie como varía el voltaje inducido de acuerdo a como se
mueve el imán. Nota: para observar una señal temporal corta conviene utilizar la función de disparo
\'unico del osciloscopio.

\subsection{Bobina e imán permanente}
¿Qué sucede cuando el campo magnético generado por un imán permanente varía dentro de
una bobina como por ejemplo cuando uno acerca o mueve un imán? Conecte una bobina al
osciloscopio y registre en función del tiempo el voltaje que se induce en la misma cuando se acerca
un imán al interior de la misma. Estudie como varía el voltaje inducido de acuerdo a como se
mueve el imán. Nota: para observar una señal temporal corta conviene utilizar la función de disparo
\'unico del osciloscopio.



\subsection{Bobina e imán permanente}
¿Qué sucede cuando el campo magnético generado por un imán permanente varía dentro de
una bobina como por ejemplo cuando uno acerca o mueve un imán? Conecte una bobina al
osciloscopio y registre en función del tiempo el voltaje que se induce en la misma cuando se acerca
un imán al interior de la misma. Estudie como varía el voltaje inducido de acuerdo a como se
mueve el imán. Nota: para observar una señal temporal corta conviene utilizar la función de disparo
\'unico del osciloscopio.

\begin{figure}
    \centering
% \begin{circuitikz}[american voltages]
% \draw
  % rotor circuit
  % (0,0) to [short, *-] (6,0)
  % to [V, l_=$\mathrm{j}{\omega}_m \underline{\psi}^s_R$] (6,2) % rotor emf
  % to [R, l_=$R_R$] (6,4) % rotor resistance
  % to [short, i_=$\underline{i}^s_R$] (5,4) % rotor current

  % stator circuit
  % (0,0) to [open, v^>=$\underline{u}^s_s$] (0,4) % stator voltage
  % to [short, *- ,i=$\underline{i}^s_s$] (1,4) % stator current
  % to [R, l=$R_s$] (3,4) % stator resistance
  % to [L, l=$L_{\sigma}$] (5,4) % leakage inductance
  % to [short, i_=$\underline{i}^s_M$] (5,3) % magnetizing current
  % to [L, l_=$L_M$] (5,0); % magnetizing inductance
% \end{circuitikz}
\caption{Figura de un circuito.}
\end{figure}
\subsection{Bobina e imán permanente}
¿Qué sucede cuando el campo magnético generado por un imán permanente varía dentro de
una bobina como por ejemplo cuando uno acerca o mueve un imán? Conecte una bobina al
osciloscopio y registre en función del tiempo el voltaje que se induce en la misma cuando se acerca
un imán al interior de la misma. Estudie como varía el voltaje inducido de acuerdo a como se
mueve el imán. Nota: para observar una señal temporal corta conviene utilizar la función de disparo
único del osciloscopio.
\subsection{Bobina e imán permanente}
¿Qué sucede cuando el campo magnético generado por un imán permanente varía dentro de
una bobina como por ejemplo cuando uno acerca o mueve un imán? Conecte una bobina al
osciloscopio y registre en función del tiempo el voltaje que se induce en la misma cuando se acerca
un imán al interior de la misma. Estudie como varía el voltaje inducido de acuerdo a como se
mueve el imán. Nota: para observar una señal temporal corta conviene utilizar la función de disparo
único del osciloscopio.
\subsection{Bobina e imán permanente}
¿Qué sucede cuando el campo magnético generado por un imán permanente varía dentro de
una bobina como por ejemplo cuando uno acerca o mueve un imán? Conecte una bobina al
osciloscopio y registre en función del tiempo el voltaje que se induce en la misma cuando se acerca
un imán al interior de la misma. Estudie como varía el voltaje inducido de acuerdo a como se
mueve el imán. Nota: para observar una señal temporal corta conviene utilizar la función de disparo
único del osciloscopio.



\subsection{Dos bobinas}

El dispositivo experimental a usar se muestra esquemáticamente en la Figura 1. El mismo
consiste en una bobina con un número N de espiras; este elemento constituye el primario del
circuito. La misma se conecta a un generador de funciones a través de una resistencia R (entre 50 y
500 Ohms). Esta sirve para limitar la corriente en la bobina y para poder medir la corriente I que
circula por el circuito primario. En general se debe evitar conectar a cualquier fuente de tensión (el
generador de funciones en este caso) elementos de poca impedancia (Z), ya que se puede
arruinar la fuente o quemar el circuito que alimenta. En un canal del osciloscopio se mide la caída
de tensión VR en la resistencia, con esto logramos tener una señal proporcional a la corriente. Se
debe tener en cuenta al diseñar el circuito que las tierras del generador y osciloscopio deben
coincidir. Una segunda bobina con un número de espiras M, se conecta al otro canal del
osciloscopio; esta segunda bobina se denomina el secundario del presente dispositivo.
Coloque una bobina dentro de la otra de modo tal que el campo magnético generado en el
primario entre dentro del bobinado del secundario. Aplique una tensión sinusoidal al circuito de la
Figura 1. Estudie como varía la amplitud de tensión inducida en el secundario como función de la
frecuencia del G.F. y luego como función de la amplitud de tensión del G.F.
Repetir la experiencia anterior colocando el núcleo de hierro en el interior de las bobinas.
Describir en forma cualitativa la relación entre las señales de corriente del primario y tensión del
secundario. Realizar esta experiencia con ondas sinusoidal y triangular. ¿Se puede decir que una sea
la derivada de la otra?
Figura 1: circuito

\subsection{Dos bobinas}

 (dependencia para distinto numero de espiras en el secundario) Transformador.
 Usando bobinas secundarias de diferente número de espiras N2, en los núcleos de hierro,
pero manteniendo las condiciones del primario constante en amplitud y frecuencia, investigue la
dependencia de V2 en función de N2. ¿Qué concluye? (es importante mantener la geometría lo más
estable posible, ¿por qué?).
El dispositivo formado por dos bobinas o espiras que comparten sus flujos, se conoce como
transformador. Mida y represente el cociente de las amplitudes (V2 / V1 ) versus (N2 / N1). Indique
en forma esquemática cómo haría para fabricar un transformador que duplique la tensión de línea y
otro que la reduzca en un factor 3.

\subsection{Bobina e imán permanente}
¿Qué sucede cuando el campo magnético generado por un imán permanente varía dentro de
una bobina como por ejemplo cuando uno acerca o mueve un imán? Conecte una bobina al
osciloscopio y registre en función del tiempo el voltaje que se induce en la misma cuando se acerca
un imán al interior de la misma. Estudie como varía el voltaje inducido de acuerdo a como se
mueve el imán. Nota: para observar una señal temporal corta conviene utilizar la función de disparo
\'unico del osciloscopio.

\subsection{Bobina e imán permanente}
¿Qué sucede cuando el campo magnético generado por un imán permanente varía dentro de
una bobina como por ejemplo cuando uno acerca o mueve un imán? Conecte una bobina al
osciloscopio y registre en función del tiempo el voltaje que se induce en la misma cuando se acerca
un imán al interior de la misma. Estudie como varía el voltaje inducido de acuerdo a como se
mueve el imán. Nota: para observar una señal temporal corta conviene utilizar la función de disparo
\'unico del osciloscopio.

\subsection{Bobina e imán permanente}
¿Qué sucede cuando el campo magnético generado por un imán permanente varía dentro de
una bobina como por ejemplo cuando uno acerca o mueve un imán? Conecte una bobina al
osciloscopio y registre en función del tiempo el voltaje que se induce en la misma cuando se acerca
un imán al interior de la misma. Estudie como varía el voltaje inducido de acuerdo a como se
mueve el imán. Nota: para observar una señal temporal corta conviene utilizar la función de disparo
\'unico del osciloscopio.

\subsection{Bobina e imán permanente}
¿Qué sucede cuando el campo magnético generado por un imán permanente varía dentro de
una bobina como por ejemplo cuando uno acerca o mueve un imán? Conecte una bobina al
osciloscopio y registre en función del tiempo el voltaje que se induce en la misma cuando se acerca
un imán al interior de la misma. Estudie como varía el voltaje inducido de acuerdo a como se
mueve el imán. Nota: para observar una señal temporal corta conviene utilizar la función de disparo
\'unico del osciloscopio.

% Referencias para esta guia
\nocite{Alonso1998, Crawford1994, Purcell1988}

\bibliographystyle{babunsrt}
\bibliography{Referencias}

\begin{sabermas}
Para saber mas haria falta leer un poco las siguientes referencias.
indeed up to 90\% of the energy is in wave modes for the lower
wavenumbers. While this results point that waves dominate the largescale
dynamics, it is also clear that they do not govern the smaller scales.
This puts theories in which eddies are not accounted for on.
\end{sabermas}


\end{document}
