\documentclass[laboratorio]{guia}

\def \practnum {5} 
\def \practica {Circuito RC: reg\'\i menes transitorio y estacionario}

\def \materia {Laboratorio de F\'\i sica II para Qu\'\i micos}
\def \periodo {2\sptext{o} cuatrimestre de 2016}
\def \profesor {Diana Skigin}
\def \website {http://materias.df.uba.ar/f2qa2016c2}
 
\usepackage{graphics}
\usepackage{amsmath}
\usepackage{amsfonts}
\usepackage{graphicx}
\usepackage{float}
\usepackage{wrapfig}
\usepackage{subfigure}
\usepackage{bm}
\usepackage{grffile}
\usepackage{color}
\usepackage{framed}
\usepackage[utf8]{inputenc}
\usepackage[T1]{fontenc}
\usepackage{lmodern} 
% definicion del entorno 'sabermas'
\makeatletter
\definecolor{shadecolor}{rgb}{0.89,0.91,0.94}
\newenvironment{sabermas}[1]{%
\vfill
\begin{shaded}
  \begin{center}
  {\textsection{Para saber m\'as}}
  \end{center}
  #1
\sf } 
{%
\end{shaded}%
}
\makeatother

\renewcommand{\vec}[1]{\ensuremath{\mathbf{#1}}}




\hyphenation{ coe-fi-cien-tes coe-fi-cien-te au-to-va-lor
              au-to-va-lo-res co-rres-pon-der pro-ble-ma 
              cual-quie-ra po-la-ri-za-cio-nes }

% \graphicspath{{./guia8/}}

\begin{document} 
\objetivo{Estudiar los comportamientos de un circuito RC en dos reg\'\i menes de operaci\'on 
    distintos: (a) transitorio y (b) estacionario. Para el transitorio, se
    propone estudiar los procesos de carga y descarga de un capacitor en un
    circuito RC. Caracterizar ambos determinando qu\'e tipo de evoluc\'on
    temporal presentan, y midiendo los tiempos caracter\'\i sticos asociados
    a cada uno de ellos. Para el estacionario, se busca determinar la respuesta
    del circuito al excitarlo con una se\~nal peri\'odica, variando la
    frecuencia de trabajo del sistema.
    \tematicas{Circuitos de corrientes variables en el tiempo, RC, carga y
        descarga de un capacitor, tiempo caracter\'\i stico, filtros pasaaltos y
    pasabajos}} 
\maketitle

\section{Introducci\'on}

Considere el circuito RC mostrado en la Figura \ref{fig:circuitoRC}, en el
cual el capacitor se encuentra completamente descargado inicialmente y la
llave S, abierta. Al cerrarse esta \'ultima, la diferencia de potencial
$V$ impuesta por la fuente genera una corriente $I$ en el circuito. Esta 
corriente tendr\'a el efecto de llevar cargas de signo opuesto a las caras 
del capacitor. Resulta intuitivo que esta corriente no ser\'a constante en el tiempo; 
en particular esperamos que la misma se anule cuando el capacitor se haya
cargado. 


Un capacitor de capacidad $C$ conectado a una fuente de tensi\'on $V$ constante 
adquiere una carga $q = C V$. Esto nos permite conocer la ca\'\i da de
potencial sobre nuestro capacitor. Por otro lado, la ecuaci\'on circuital para
el circuito RC resulta simplemente:

\begin{equation}
    V = RI + \frac{q}{C},
\end{equation}
donde tanto la corriente $I$ como la carga $q$ est\'an variando instante a
instante, es decir que $I \equiv I(t)$ y $q \equiv q(t)$. Recordemos, por otro
lado, que tanto la tensi\'on $V$ de la fuente, la resistencia $R$ del resistor
y la capacidad $C$ del capacitor son constantes, dado que describen propiedades de
cada uno de dichos elementos. Empleando ahora la definici\'on de corriente,

$$ I = \frac{dq}{dt}, $$
podemos reescribir la \'ultima ecuaci\'on en t\'erminos de una \'unica
funci\'on inc\'ognita, ya sea $q(t)$ o $I(t)$. Vamos a elegir reexpresarla en
funci\'on de $q(t)$, de lo que se obtiene

\begin{equation}
    V = R \frac{dq}{dt}(t) + \frac{1}{C} q(t).
\end{equation}

Esta ecuaci\'on es una ecuaci\'on diferencial ordinaria de orden 1 para $q(t)$, 
cuya soluci\'on nos dar\'a la evoluci\'on temporal (desde un instante inicial
dado) de la carga en el capacitor. 
Para resolverla, debemos especificar adem\'as una condici\'on inicial para la 
carga $q(t)$ en el capacitor. Dado que estamos considerando el caso en el que
el mismo 

\begin{figure}[t!]
    \centering
    \begin{tikzpicture}
        % -- Circuito RC -- % \draw[step=0.5, very thin, black!20] (-1, -0.5) grid (6, 2.5);
        \path (0, 0) coordinate (ref_gnd);
        \draw
          (ref_gnd) to[battery1=\(V\)] ++(0,2)
                    to[R=\(R\)] ++(3,0) 
                    to[C=\(C\)] ++(0,-2) 
          -- (ref_gnd);
    \end{tikzpicture}
    \vspace{0.5cm}
    \caption{Esquema del circuito RC empleado.}
    \label{fig:circuitoRC}
\end{figure}

\section{Carga y descarga de un capacitor}

En esta primera etapa se estudia el proceso de carga y descarga del capacitor.




\section{An\'alisis exploratorio semi-cuantitativo}

Equipamiento b\'asico recomendado: Una bandeja de vidrio o acr\'\i lico transparente, de aproximadamente $(30 \times 20 \times 4)$~cm$^3$. Una fuente de tensi\'on continua de 5-15 V. Un volt\'\i metro. Placas met\'alicas (de cobre, bronce, aluminio) para emplear como electrodos.

Utilizando un dispositivo experimental similar al ilustrado en la Fig. 1: 

\begin{enumerate}
    \item Determine las l\'\i neas equipotenciales en la zona entre los electrodos.
    \item Para la misma configuraci\'on anterior, coloque un conductor entre los electrodos y determine las l\'\i neas equipotenciales de este nuevo arreglo (ver Fig. 2). En particular, estudie la forma de las  l\'\i neas equipotenciales alrededor del conductor. ?`C\'omo deber\'\i an ser las l\'\i neas equipotenciales dentro del mismo?  
    \item Repita las mediciones reemplazando ahora el conductor por un aislante. 
\end{enumerate}

\begin{sabermas} Para saber mas haria falta leer un poco las siguientes
    referencias.  indeed up to 90\% of the energy is in wave modes for the
    lower wavenumbers. While this results point that waves dominate the
    largescale dynamics, it is also clear that they do not govern the smaller
    scales.  This puts theories in which eddies are not accounted for on.
\end{sabermas}

% \bibliographystyle{unsrt} \bibliography{biblio}

\end{document}
