\documentclass[laboratorio]{guia}

\def \practnum {1} 
\def \practica {Electrost\'atica}

\def \materia {Laboratorio de F\'\i sica II para Qu\'\i micos}
\def \periodo {2\sptext{o} cuatrimestre de 2016}
\def \profesor {Diana Skigin}
\def \website {http://materias.df.uba.ar/f2qa2016c2}
 
\usepackage{graphics}
\usepackage{amsmath}
\usepackage{amsfonts}
\usepackage{graphicx}
\usepackage{float}
\usepackage{wrapfig}
\usepackage{subfigure}
\usepackage{bm}
\usepackage{grffile}
\usepackage{color}
\usepackage{framed}
\usepackage[utf8]{inputenc}
\usepackage[T1]{fontenc}
\usepackage{lmodern} 
% definicion del entorno 'sabermas'
\makeatletter
\definecolor{shadecolor}{rgb}{0.89,0.91,0.94}
\newenvironment{sabermas}[1]{%
\vfill
\begin{shaded}
  \begin{center}
  {\textsection{Para saber m\'as}}
  \end{center}
  #1
\sf } 
{%
\end{shaded}%
}
\makeatother

\renewcommand{\vec}[1]{\ensuremath{\mathbf{#1}}}




\hyphenation{ coe-fi-cien-tes coe-fi-cien-te au-to-va-lor
              au-to-va-lo-res co-rres-pon-der pro-ble-ma 
              cual-quie-ra po-la-ri-za-cio-nes }

% \graphicspath{{./guia8/}}

\begin{document} 
\objetivo{Determinar el mapa de l\'\i neas o superficies equipotenciales para
    distintas configuraciones de electrodos conectados a una fuente de baja
    tensi\'on e inmersos en un medio l\'\i quido poco conductor.
    \tematicas{Electrost\'atica, potencial electrost\'atico, campo el\'ectrico,
conductores y diel\'ectricos.}} 
\maketitle

\section{Introducci\'on}

El campo el\'ectrico en un dado punto del espacio est\'a relacionado con la fuerza
el\'ectrica que se ejerce sobre una carga de prueba $q$ colocada en ese punto. Si
en el punto de coordenadas $(x,y)$ existe un campo el\'ectrico $\vec{E}(x,y)$,
sobre la carga $q$, colocada en ese punto se ejerce una fuerza $\vec{F}(x,y)$.
Seg\'un la definici\'on de campo el\'ectrico tenemos:

\begin{equation} 
\vec{F}(x,y) = q \: \vec{E}(x,y).  
\end{equation}

Como la fuerza $\vec{F}$ es un vector y la carga el\'ectrica $q$ un escalar, resulta claro que el campo el\'ectrico local $\vec{E}$ es tambi\'en
un vector. Por su parte, el potencial el\'ectrico $V$ est\'a relacionado con el
trabajo $W$ que debemos realizar para llevar una carga de un punto a otro; m\'as
precisamente, el cambio en el potencial entre dos puntos 1 y 2 ser\'a: 

\begin{equation}
    \Delta V_{12} = \frac{W_{12}}{q},
\end{equation}
Aqu\'i $W_{12}$ es el trabajo que tenemos que realizar para llevar la
carga $q$ desde el punto 1 al punto 2. Como el trabajo es una magnitud escalar, el
potencial tambi\'en lo es. En t\'erminos del campo el\'ectrico, la variaci\'on de potencial entre dos puntos del espacio separados por una distancia infinitesimal $d\vec{l}$ viene dada por:

\begin{equation}
    dV_{12} \equiv V_2 - V_1 = - \frac{dW}{q} = - \frac{1}{q} \vec{F}(x,y)
    \cdot d\vec{l} = - \vec{E} \cdot d\vec{l}.
\end{equation}
Por lo tanto, las componentes del campo el\'ectrico pueden expresarse en funci\'on del potencial el\'ectrico:
\begin{equation}
    \vec{E} = - \vec{\nabla} V,
\end{equation}
expresi\'on que resulta v\'alida en cualquier sistema de coordenadas. 

\section{An\'alisis exploratorio semi-cuantitativo}

Equipamiento b\'asico recomendado: Una bandeja de vidrio o acr\'\i lico transparente, de aproximadamente $(30 \times 20 \times 4)$~cm$^3$. Una fuente de tensi\'on continua de 5-15 V. Un volt\'\i metro. Placas met\'alicas (de cobre, bronce, aluminio) para emplear como electrodos.

Utilizando un dispositivo experimental similar al ilustrado en la Fig. 1: 

\begin{enumerate}
    \item Determine las l\'\i neas equipotenciales en la zona entre los electrodos.
    \item Para la misma configuraci\'on anterior, coloque un conductor entre los electrodos y determine las l\'\i neas equipotenciales de este nuevo arreglo (ver Fig. 2). En particular, estudie la forma de las  l\'\i neas equipotenciales alrededor del conductor. ?`C\'omo deber\'\i an ser las l\'\i neas equipotenciales dentro del mismo?  
    \item Repita las mediciones reemplazando ahora el conductor por un aislante. 
\end{enumerate}

\begin{sabermas} Para saber mas haria falta leer un poco las siguientes
    referencias.  indeed up to 90\% of the energy is in wave modes for the
    lower wavenumbers. While this results point that waves dominate the
    largescale dynamics, it is also clear that they do not govern the smaller
    scales.  This puts theories in which eddies are not accounted for on.
\end{sabermas}

% \bibliographystyle{unsrt} \bibliography{biblio}

\end{document}
